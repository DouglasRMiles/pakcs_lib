\section{ERD2Curry: A Tool to Generate Programs from ER Specifications}
\label{sec-erd2curry}

ERD2Curry\index{ERD2Curry}\index{database programming}
is a tool to generate Curry code to access and manipulate data
persistently stored in relational databases.
The Curry code is generated from a description of the logical model
of the database in form of an
entity relationship diagram.\index{entity relationship diagram}
The idea of this tool is described in detail in
\cite{BrasselHanusMueller08PADL}.
Thus, we describe only the basic steps to use this tool.

\subsection{Installation}

The current implementation of ERD2Curry is a package
managed by the Curry Package Manager CPM
(see also Section~\ref{sec-cpm}).
Thus, to install the newest version of ERD2Curry, use the following commands:
%
\begin{curry}
> cpm update
> cpm installapp ertools
\end{curry}
%
This downloads the newest package, compiles it, and places
the executable \code{erd2curry} into the directory \code{\$HOME/.cpm/bin}.
Hence it is recommended to add this directory to your path
in order to execute CurryVerify as described below.

\subsection{Basic Usage}

If one creates an entity relationship diagram (ERD)
with the Umbrello UML Modeller, one has to store its
XML description in XMI format (as offered by Umbrello)
in a file, e.g., \ccode{myerd.xmi}.
This description can be compiled into a Curry program by the
command\pindex{curry erd2curry}\pindex{erd2curry}
\begin{curry}
erd2curry -x myerd.xmi
\end{curry}
If \code{MyData} is the name of the ERD, the Curry program file
\ccode{MyData.curry} is generated containing all the necessary
database access code as described in \cite{BrasselHanusMueller08PADL}.
In addition to the generated Curry program file,
two auxiliary program files
\code{ERDGeneric.curry} and \code{KeyDatabase.curry}
are created in the same directory.

%
\begin{figure}[t]
\begin{curry}
import Database.ERD

blogERD :: ERD
blogERD =
 ERD "Blog" 
     [Entity "Entry" 
          [Attribute "Title"  (StringDom Nothing) Unique False, 
           Attribute "Text"   (StringDom Nothing) NoKey  False,
           Attribute "Author" (StringDom Nothing) NoKey  False,
           Attribute "Date"   (DateDom   Nothing) NoKey  False],
      Entity "Comment"
        [Attribute "Text"   (StringDom Nothing) NoKey False,
         Attribute "Author" (StringDom Nothing) NoKey False,
         Attribute "Date"   (DateDom   Nothing) NoKey False],
      Entity "Tag"
        [Attribute "Name" (StringDom Nothing) Unique False]
     ]
     [Relationship "Commenting"
       [REnd "Entry"   "commentsOn"    (Exactly 1),
        REnd "Comment" "isCommentedBy" (Between 0 Infinite)],
      Relationship "Tagging"
       [REnd "Entry" "tags" (Between 0 Infinite),
        REnd "Tag" "tagged" (Between 0 Infinite)]
     ]
\end{curry}
\caption{The Curry program \code{BlogERD.curry}\label{fig:blogerdcurry}}
\end{figure}

If one does not want to use the Umbrello UML Modeller,
which might be the preferred method since the interface to the
Umbrello UML Modeller is no longer actively supported,
one can also define an ERD in a Curry program as a (exported!)
top-level operation of type \code{ERD}
(w.r.t.\ the type definition given in the library
\code{\cyshome/lib/Database/ERD.curry}).
%
The directory \code{examples} in the package \code{ertools}\footnote{%
If you installed ERD2Curry as described above,
the downloaded \code{ertools} package is located in the directory
\code{\$HOME/.cpm/bin_packages/ertools}.}
contains two examples for such ERD program files:
\begin{description}
\item[\code{BlogERD.curry}:]
This is a simple ERD model for a blog with entries, comments,
and tags.
\item[\code{UniERD.curry}:]
This is an ERD model for university lectures as
presented in the paper \cite{BrasselHanusMueller08PADL}.
\end{description}
%
Figure~\ref{fig:blogerdcurry} shows the ER specification
stored in the Curry program file \ccode{BlogERD.curry}.
This ER specification can be compiled into a Curry program by the
command\pindex{erd2curry}
\begin{curry}
erd2curry BlogERD.curry
\end{curry}
%
There is also the possibility to visualize an ER specification
as a graph with the graph visualization program \code{dotty}
(for this purpose, it might be necessary to adapt the definition
of \code{dotviewcommand} in your \ccode{\curryrc} file,
see Section~\ref{sec-customization},
according to your local environment).
The visualization can be performed by the command
\begin{curry}
erd2curry -v BlogERD.curry
\end{curry}

