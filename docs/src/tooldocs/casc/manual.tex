\section{\texttt{curry style}: A Style Checker for Curry Programs}
\label{sec-casc}

CASC\index{CASC}\index{style checking}
is a tool to check the formatting style of Curry programs.
The preferred style for writing Curry programs,
which is partially checked by this tool,
is described in a separate web page\footnote{%
\url{http://www.informatik.uni-kiel.de/~pakcs/CurryStyleGuide.html}}
Currently, CASC only checks a few formatting rules, like
line lengths or
indentation of \code{if-then-else}, but the kind of checks
performed by CASC will be extended in the future.

\subsection{Basic Usage}

To check the style of some Curry program stored
in the file \code{prog.curry},
one can invoke the style checker by
the command\pindex{curry style}\pindex{style}
%
\begin{curry}
curry style prog
\end{curry}
%
After processing the program, a list of all positions
with stylistic errors is printed.


\subsection{Configuration}

CASC can be configured so that not all stylistic rules are checked.
For this purpose, one should copy the global configuration file of CASC,
which is stored in \code{\cyshome/currytools/casc/cascrc}
(where \cyshome is the installation directory of \CYS),
into the home directory under the name \ccode{.cascrc}.
Then one can configure this file according to your own preferences,
which are described in this file.

%  LocalWords:  CASC
