\section{Preprocessing FlatCurry Files}
\label{sec-pakcspp}

After the invocation of the Curry front end to parse
Curry programs and translate them into the intermediate FlatCurry
representation, one can apply transformations on the FlatCurry files
before they are passed to the back end which translates
the FlatCurry files into Prolog code.
These transformation are invoked by the FlatCurry preprocessor
\code{pakcs/bin/fycpp}.\pindex{fcypp}
Currently, only the FlatCurry file corresponding to the main module
can be transformed.

A transformation can be specified as follows:
\begin{enumerate}
\item {\bf Options to \code{pakcs/bin/fcypp}:}
\begin{description}
\item[\fbox{\code{--fpopt}}]\pindex{-fpopt}
Apply functional pattern optimization
(see \code{pakcs/tools/optimize/NonStrictOpt.curry} for details).

\item[\fbox{\code{--compact}}]\pindex{--compact}
Apply code compactification after parsing, i.e., transform the main
module and all its imported into one module and delete all
non-accessible functions.

\item[\fbox{\code{--compactexport}}]
Similar to \code{--compact} but delete all functions that are not accessible
from the exported functions of the main module.

\item[\fbox{\code{--compactmain:f}}]
Similar to \code{--compact} but delete all functions that are not accessible
from the function \ccode{f} of the main module.

\item[\fbox{\code{--fcypp cmd}}]\pindex{--fcypp}
Apply command \code{cmd} to the main module after parsing. This is useful to
integrate your own transformation into the compilation process.
Note that the command \ccode{cmd prog} should perform a transformation
on the FlatCurry file \code{prog.fcy}, i.e., it replaces the FlatCurry
file by a new one.
\end{description}

\item {\bf Setting the environment variable \code{FCYPP}:}\pindex{FCYPP}\\
For instance, setting \code{FCYPP} by
\begin{curry}
export FCYPP="--fpopt"
\end{curry}
will apply the functional pattern optimization if programs are compiled
and loaded in the \CYS programming environment.

\item {\bf Putting options into the source code:}\pindex{PAKCS_OPTION_FCYPP}\\
If the source code contains a line with a comment of the form (the comment
must start at the beginning of the line)
\begin{curry}
{-# PAKCS_OPTION_FCYPP <options> #-}
\end{curry}
then the transformations specified by \code{<options>} are applied after
translating the source code into FlatCurry code. For instance,
the functional pattern optimization can be set by the comment
\begin{curry}
{-# PAKCS_OPTION_FCYPP --fpopt #-}
\end{curry}
in the source code. Note that this comment must be in a single line 
of the source program. If there are multiple lines containing such comments,
only the first one will be considered.
\end{enumerate}
\paragraph{Multiple options:}
Note that an arbitrary number of transformations can be specified
by the methods described above.
If several specifications for preprocessing FlatCurry files are used,
they are executed in the following order:
\begin{enumerate}
\item all transformations specified by the environemnt variable
\code{FCYPP} (from left to right)
\item all transformations specified as command line options of \code{fcypp}
   (from left to right)
\item all transformations specified by a comment line in the source code
   (from left to right)
\end{enumerate}


%%% Local Variables: 
%%% mode: latex
%%% TeX-master: "manual"
%%% End: 
