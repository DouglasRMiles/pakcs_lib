\section{Overview of \CYS}

\subsection{General Use}
\label{sec-general}

This version of \CYS has been tested on Sun Solaris, Linux, and Mac OS X
systems. In principle, it should be also executable on other
platforms on which a Prolog system like SICStus-Prolog or SWI-Prolog exists
(see the file \code{INSTALL.html} in the \CYS directory
for a description of the necessary software to install \CYS).

All executable files required to use the different components
of \CYS are stored in the directory \code{\cyshome/bin}
(where \cyshome is the installation directory of the complete
\CYS installation). You should add this directory
to your path (e.g., by the \code{bash} command
\ccode{export PATH=\cyshome/bin:\$PATH}).

The source code of the Curry program
must be stored in a file with the suffix \ccode{.curry},
e.g., \code{prog.curry}. 
Literate programs must be stored in files with the extension \ccode{.lcurry}.
%They are automatically converted into corresponding
%\ccode{.curry} files by deleting all lines not starting 
%with \ccode{>} and removing the prefix \ccode{> } of the
%remaining lines.

Since the translation of Curry programs with \CYS creates
some auxiliary files (see Section~\ref{sec-auxfiles} for details),
you need write permission
in the directory where you have stored your Curry programs.
The auxiliary files for all Curry programs in the current
directory can be deleted by the command\pindex{cleancurry}
\begin{curry}
cleancurry
\end{curry}
(this is a shell script stored in the \code{bin} directory of the
\CYS installation, see above).
The command
\begin{curry}
cleancurry -r
\end{curry}
also deletes the auxiliary files in all subdirectories.



\subsection{Restrictions}
\label{sec-restrictions}

There are a few minor restrictions on Curry programs
when they are processed with \CYS:
\begin{itemize}
\item
\index{singleton variables}\index{variables!singleton}
\emph{Singleton pattern variables}, i.e., variables that occur only once
in a rule, should be denoted as an anonymous variable \ccode{_},
otherwise the parser will print a warning since this is a
typical source of programming errors.
\item
\CYS translates all \emph{local declarations} into global functions with
additional arguments (``lambda lifting'', see Appendix~D of the
Curry language report).
Thus, in the compiled target code, the definition of
functions with local declarations look different from
their original definition (in order to see the result
of this transformation, you can use the \cb, see
Section~\ref{sec-currybrowser}).
\item \index{tabulator stops}
Tabulator stops instead of blank spaces in source files are
interpreted as stops at columns 9, 17, 25, 33, and so on.
In general, tabulator stops should be avoided in source programs.
\item
Since PAKCS compiles Curry programs into Prolog programs,
non-deterministic computations are treated as in Prolog
by a backtracking strategy, which is known to be incomplete.
Thus, the order of rules could influence the ability
to find solutions for a given goal.
\item
Threads created by a concurrent conjunction are not executed
in a fair manner (usually, threads corresponding to leftmost constraints
are executed with higher priority).
\item
Encapsulated search\index{encapsulated search}: In order
to allow the integration of non-deterministic computations
in programs performing I/O at the top-level, \CYS supports
the search operators \code{findall}\pindex{findall}
and \code{findfirst}\pindex{findfirst}.
In contrast to the general definition of encapsulated search
\cite{HanusSteiner98PLILP}, the current implementation suspends
the evaluation of \code{findall} and \code{findfirst}
until the argument does not contain unbound global variables.
Moreover, the evaluation of \code{findall} is strict,
i.e., it computes all solutions before returning the
complete list of solutions.
It is recommended to use the system module \code{AllSolutions}
for encapsulating search.
\item
There is currently no general connection to external constraint solvers.
However, the \CYS compiler provides constraint
solvers for arithmetic and finite domain constraints
(see Appendix~\ref{sec:libraries}).
\end{itemize}

% Layout rule:
% (from Sergio's email of June 2, 1998)
%This is the general rule.  There are two kinds of syntactic
%constructs that rely on the offside rule.  One kind has a keyword
%indicating the end of the construct.  "let ... in" is the only
%representative of this kind.  Upon recognition of the keyword
%"in", all the constructs relying on the offide rule nested within
%the "let...in" are closed.  The other kind has no closing keyword.
%"where" and "choice" are the only constructs of this kind.
%Constructs of this kind can be closed only by indentation.  Any
%line, including a comment, indented less that the construct
%terminates it.  The indentation of "where", "choice" and "let" is
%the indentation of the first token following the keyword of the
%construct.
%



\subsection{Modules in \CYS}
\label{sec-modules}

The current implementation of \CYS supports only flat module names,
i.e., the notation \code{Dir.Mod.f} is not supported.\index{modules}
In order to allow the structuring of modules in different directories,
\CYS searches for imported modules in various directories.
By default, imported modules are searched in the directory
of the main program and the system module directories
\ccode{\cyshome/lib} and \ccode{\cyshome/lib/meta}.
This search path can be extended
by setting the environment variable \code{CURRYPATH}\pindex{CURRYPATH}
(which can be also set in a \CYS session by the command
\ccode{:set path}\pindex{path}\pindex{:set path},
see below)
to a list of directory names separated by colons (\ccode{:}).
In addition, a local standard search path
can be defined in the \ccode{.pakcsrc} file
(see Section~\ref{sec-customization}).
Thus, modules to be loaded are searched in the following
directories (in this order, i.e., the first occurrence of a module file
in this search path is imported):
\begin{enumerate}
\item Current working directory (\ccode{.}) or directory prefix
of the main module (e.g., directory \ccode{/home/joe/curryprogs}
if one loads the Curry program \ccode{/home/joe/curryprogs/main}).
\item The directories enumerated in the environment variable \code{CURRYPATH}.
\item The directories enumerated in the \ccode{.pakcsrc} variable
      \ccode{libraries}.
\item The directories \ccode{\cyshome/lib} and \ccode{\cyshome/lib/meta}.
\end{enumerate}
Note that the standard prelude (\code{\cyshome/lib/Prelude.curry})
will be always implicitly imported to all modules if a module
does not contain an explicit import declaration for the module
\code{Prelude}.


%%% Local Variables: 
%%% mode: latex
%%% TeX-master: "manual"
%%% End: 
