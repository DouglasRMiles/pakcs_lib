\section{Libraries of the \CYS Distribution}
\label{sec:libraries}

{\setlength{\parindent}{0.0cm}

The \CYS distribution comes with an extensive collection
of libraries for application programming.
The libraries for arithmetic constraints over real numbers,
finite domain constraints,
ports for concurrent and distributed programming, and
meta-programming by representing Curry programs in Curry
are described in the following subsection in more detail.
The complete set of libraries with all exported types and functions
are described in the further subsections.
For a more detailed online documentation of all libraries of \CYS,
see \url{http://www.informatik.uni-kiel.de/~pakcs/lib/index.html}.

\subsection{Constraints, Ports, Meta-Programming}

\subsubsection{Arithmetic Constraints}

The primitive entities for the use of arithmetic constraints
are defined in the system module \code{CLPR}
(cf.\ Section~\ref{sec-modules}), i.e., in order to use them,
the program must contain the import declaration
\begin{curry}
import CLPR
\end{curry}
Floating point arithmetic is supported in \CYS
via arithmetic constraints, i.e., the equational constraint
\ccode{2.3 +.~x =:= 5.5} is solved by binding \code{x} to \code{3.2}
(rather than suspending the evaluation of the addition,
as in corresponding constraints on integers like
\ccode{3+x=:=5}). All operations related to
floating point numbers are suffixed by \ccode{.}.
The following functions and constraints on floating point
numbers are supported in \CYS:
\begin{description}
\item[\code{(+.)   :: Float -> Float -> Float}]~\\
Addition on floating point numbers.
\item[\code{(-.)   :: Float -> Float -> Float}]~\\
Subtraction on floating point numbers.
\item[\code{(*.)   :: Float -> Float -> Float}]~\\
Multiplication on floating point numbers.
\item[\code{(/.)   :: Float -> Float -> Float}]~\\
Division on floating point numbers.
\item[\code{(<.)   :: Float -> Float -> Bool}]~\\
Comparing two floating point numbers with the ``less than'' relation.
\item[\code{(>.)   :: Float -> Float -> Bool}]~\\
Comparing two floating point numbers with the ``greater than'' relation.
\item[\code{(<=.)  :: Float -> Float -> Bool}]~\\
Comparing two floating point numbers with the ``less than or equal'' relation.
\item[\code{(>=.)  :: Float -> Float -> Bool}]~\\
Comparing two floating point numbers with the ``greater than or equal''
relation.
\item[\code{i2f    :: Int -> Float}]~\\
Converting an integer number into a floating point number.
\end{description}
As an example, consider a constraint \code{mortgage}
which relates the principal \code{p},
the lifetime of the mortgage in months \code{t},
the monthly interest rate \code{ir},
the monthly repayment \code{r},
and the outstanding balance at the end of the lifetime \code{b}.
The financial calculations
can be defined by the following two rules in Curry (the second rule
describes the repeated accumulation of the interest):
\begin{curry}
import CLPR

mortgage p t ir r b | t >. 0.0 \& t <=. 1.0  --lifetime not more than 1 month?
                    =  b =:= p *. (1.0 +. t *. ir) -. t*.r $\listline$
mortgage p t ir r b | t >. 1.0               --lifetime more than 1 month?
                    =  mortgage (p *. (1.0+.ir)-.r) (t-.1.0) ir r b
\end{curry}
Then we can calculate the monthly payment for paying back
a loan of \$100,000 in 15 years with a monthly interest rate of 1\%
by solving the goal
\begin{curry}
mortgage 100000.0 180.0 0.01 r 0.0
\end{curry}
which yields the solution \code{r=1200.17}.

Note that only linear arithmetic equalities or inequalities
are solved by the constraint solver. Non-linear constraints
like \ccode{x *.~x =:= 4.0} are suspended until they become
linear.


\subsubsection{Finite Domain Constraints}

Finite domain constraints are constraints where all variables
can only take a finite number of possible values.
For simplicity, the domain of finite domain variables are
identified with a subset of the integers, i.e., the type
of a finite domain variable is \code{Int}. The arithmetic
operations related to finite domain variables are suffixed by \ccode{\#}.
The following functions and constraints for finite domain constraint solving
are currently supported in \CYS:\footnote{Note that
this library is based on the corresponding library of SICStus-Prolog
but does not implement the complete functionality of the SICStus-Prolog library.
However, using the \CYS interface for external functions (see
Appendix~\ref{sec-external-functions}), it is relatively
easy to provide the complete functionality.}

\begin{description}
\item[\code{domain :: [Int] -> Int -> Int -> Bool}]~\\
The constraint \ccode{domain [$x_1,\ldots,x_n$] $l$ $u$}
is satisfied if the domain of all variables $x_i$ is the interval $[l,u]$.
\item[\code{(+\#)   :: Int -> Int -> Int}]~\\
Addition on finite domain values.
\item[\code{(-\#)   :: Int -> Int -> Int}]~\\
Subtraction on finite domain values.
\item[\code{(*\#)   :: Int -> Int -> Int}]~\\
Multiplication on finite domain values.
\item[\code{(=\#)   :: Int -> Int -> Bool}]~\\
Equality of finite domain values.
\item[\code{(/=\#)  :: Int -> Int -> Bool}]~\\
Disequality of finite domain values.
\item[\code{(<\#)   :: Int -> Int -> Bool}]~\\
``less than'' relation on finite domain values.
\item[\code{(<=\#)  :: Int -> Int -> Bool}]~\\
``less than or equal'' relation on finite domain values.
\item[\code{(>\#)   :: Int -> Int -> Bool}]~\\
``greater than'' relation on finite domain values.
\item[\code{(>=\#)  :: Int -> Int -> Bool}]~\\
``greater than or equal'' relation on finite domain values.
\item[\code{sum :: [Int] -> (Int -> Int -> Bool) -> Int -> Bool}]~\\
The constraint \ccode{sum [$x_1,\ldots,x_n$] $op$ $x$}
is satisfied if all $x_1+\cdots + x_n \mathrel{op} x$ is satisfied,
where $op$ is one of the above finite domain constraint relations
(e.g., \ccode{=\#}).
\item[\code{scalar_product :: [Int] -> [Int] -> (Int -> Int -> Bool) -> Int -> Bool}]~\\
The constraint \ccode{scalar_product [$c_1,\ldots,c_n$] [$x_1,\ldots,x_n$] $op$ $x$}
is satisfied if all $c_1 x_1+\cdots + c_n x_n \mathrel{op} x$ is satisfied,
where $op$ is one of the above finite domain constraint relations.
\item[\code{count :: Int -> [Int] -> (Int -> Int -> Bool) -> Int -> Bool}]~\\
The constraint \ccode{count $k$ [$x_1,\ldots,x_n$] $op$ $x$}
is satisfied if all $k \mathrel{op} x$ is satisfied,
where $n$ is the number of the $x_i$ that are equal to $k$ and
$op$ is one of the above finite domain constraint relations.
\item[\code{allDifferent :: [Int] -> Bool}]~\\
The constraint \ccode{allDifferent [$x_1,\ldots,x_n$]}
is satisfied if all $x_i$ have pairwise different values.
\item[\code{labeling :: [LabelingOption] -> [Int] -> Bool}]~\\
The constraint \ccode{labeling $os$ [$x_1,\ldots,x_n$]}
non-deterministically instantiates all $x_i$ to the values
of their domain according to the options $os$ (see the module documentation
for further details about these options).
\end{description}
These entities are defined in the system module \code{CLPFD}
(cf.\ Section~\ref{sec-modules}), i.e., in order to use it,
the program must contain the import declaration
\begin{curry}
import CLPFD
\end{curry}
As an example, consider the classical \ccode{send+more=money} problem
where each letter must be replaced by a different digit such that this
equation is valid and there are no leading zeros.
The usual way to solve finite domain constraint problems
is to specify the domain of the involved variables followed
by a specification of the constraints and the labeling
of the constraint variables in order to start the search for solutions.
Thus, the \ccode{send+more=money} problem can be solved as follows:
\begin{curry}
import CLPFD

smm l =
        l =:= [s,e,n,d,m,o,r,y] &
        domain l 0 9 &
        s ># 0 &
        m ># 0 &
        allDifferent l  &
                         1000 *# s +# 100 *# e +# 10 *# n +# d
        +#               1000 *# m +# 100 *# o +# 10 *# r +# e
        =# 10000 *# m +# 1000 *# o +# 100 *# n +# 10 *# e +# y &
        labeling [FirstFail] l
        where s,e,n,d,m,o,r,y free
\end{curry}
Then we can solve this problem by evaluating the goal
\ccode{smm [s,e,n,d,m,o,r,y]} which yields the unique solution
\code{\{s=9,e=5,n=6,d=7,m=1,o=0,r=8,y=2\}}.


\subsubsection{Ports: Distributed Programming in Curry}
\label{sec-ports}

To support the development of concurrent and distributed applications,
\CYS supports internal and external ports\index{ports} as
described in \cite{Hanus99PPDP}.
Since \cite{Hanus99PPDP} contains a detailed description of this
concept together with various programming examples, we only summarize here
the functions and constraints supported for ports in \CYS.

The basic datatypes, functions, and constraints for ports
are defined in the system module \code{Ports}
(cf.\ Section~\ref{sec-modules}), i.e., in order to use ports,
the program must contain the import declaration
\begin{curry}
import Ports
\end{curry}
This declaration includes the following entities in the program:
\begin{description}
\item[\code{Port a}\pindex{Port}]~\\
This is the datatype of a port to which one can send messages of type \code{a}.

\item[\code{openPort :: Port a -> [a] -> Bool}]~\\
The constraint \ccode{openPort p s}\pindex{openPort}
establishes a new \emph{internal port}
\code{p} with an associated message stream \code{s}. \code{p} and \code{s} must be
unbound variables,
otherwise the constraint fails (and causes a runtime error).

\item[\code{send :: a -> Port a -> Bool}]~\\
The constraint \ccode{send m p}\pindex{send}
is satisfied if \code{p} is constrained
to contain the message \code{m}, i.e., \code{m} will be sent to the port
\code{p} so that it appears in the corresponding stream.

\item[\code{doSend :: a -> Port a -> IO ()}]~\\
The I/O action \ccode{doSend m p}\pindex{doSend} solves the constraint
\ccode{send m p} and returns nothing.

\item[\code{openNamedPort :: String -> IO [a]}]~\\
The I/O action \ccode{openNamedPort n}\pindex{openNamedPort}
opens a new \emph{external port} with
symbolic name \code{n} and returns the associated stream of messages.

\item[\code{connectPort :: String -> IO (Port a)}]~\\
The I/O action \ccode{connectPort n}\pindex{connectPort}
returns a port with symbolic name
\code{n} (i.e., \code{n} must have the form ``\emph{portname@machine})
to which one can send messages by the \code{send} constraint.
Currently, no dynamic type checking is done for external ports,
i.e., sending messages of the wrong type to a port might lead to
a failure of the receiver.
\end{description}

\paragraph{Restrictions:}
Every expression, possibly containing logical variables, can be sent to
a port. However, as discussed in \cite{Hanus99PPDP},
port communication is strict, i.e., the expression is
evaluated to normal form before sending it by the
constraint \code{send}. Furthermore, if messages containing
logical variables are sent to \emph{external ports},
the behavior is as follows:
\begin{enumerate}
\item The sender waits until all logical variables in the message
have been bound by the receiver.
\item The binding of a logical variable received by a process
is sent back to the sender of this logical variable only if
it is bound to a \emph{ground} term, i.e., as long as the binding contains
logical variables, the sender is not informed about the binding
and, therefore, the sender waits.
\end{enumerate}

\paragraph{External ports on local machines:}
The implementation of external ports assumes that the
host machine running the application is connected to the Internet
(i.e., it uses the standard IP address of the host machine
for message sending). If this is not the case and the application
should be tested by using external ports only on the local host
without a connection to the Internet,
the environment variable \ccode{PAKCS_LOCALHOST}\pindex{PAKCS_LOCALHOST}
must be set to \ccode{yes}
\emph{before \CYS is started}.
In this case, the IP address \code{127.0.0.1} and the hostname
\ccode{localhost} are used for identifying the local machine.

\paragraph{Selection of Unix sockets for external ports:}
The implementation of ports uses sockets to communicate
messages sent to external ports.
Thus, if a Curry program uses the
I/O action \code{openNamedPort}\pindex{openNamedPort}
to establish an externally visible server,
\CYS selects a Unix socket for the port communication.
Usually, a free socket is selected by the operating system.
If the socket number should be fixed in an application (e.g.,
because of the use of firewalls\index{firewall} that allow only
communication over particular sockets), then one
can set the environment variable \ccode{PAKCS_SOCKET}\pindex{PAKCS_SOCKET}
to a distinguished socket number before \CYS is started.
This has the effect that \CYS uses only this socket
number for communication (even for several external ports
used in the same application program).

\paragraph{Debugging:}
To debug distributed systems,
it is sometimes helpful to see all messages sent to external ports.
This is supported by the environment variable
\ccode{PAKCS_TRACEPORTS}.\pindex{PAKCS_TRACEPORTS}
If this variable is set to \ccode{yes}
\emph{before \CYS is started}, then all
connections to external ports and all
messages sent and received on external ports are
printed on the standard error stream.


\subsubsection{AbstractCurry and FlatCurry: Meta-Programming in Curry}
\label{sec-flatcurry}

\index{AbstractCurry}
\index{FlatCurry}
To support meta-programming, i.e., the manipulation of Curry programs
in Curry, there are system modules \code{AbstractCurry.Types}
and \code{FlatCurry.Types}
which define datatypes for the representation
of Curry programs.
\code{AbstractCurry.Types} is a more direct representation of a Curry program,
whereas \code{FlatCurry.Types} is a simplified representation
where local function definitions are replaced by global definitions
(i.e., lambda lifting has been performed) and pattern matching
is translated into explicit case/or expressions.
Thus, \code{FlatCurry.Types} can be used for more back-end oriented
program manipulations (or, for writing new back ends for Curry),
whereas \code{AbstractCurry.Types} is intended for manipulations of
programs that are more oriented towards the source program.

There are predefined I/O actions to read AbstractCurry and
FlatCurry programs: \code{AbstractCurry.Files.readCurry}\pindex{readCurry})
and \code{FlatCurry.Files.readFlatCurry}\pindex{readFlatCurry}).
These actions parse the corresponding source program and return
a data term representing this program (according to the definitions
in the modules \code{AbstractCurry.Types} and \code{FlatCurry.Types}).

Since all datatypes are explained in detail in these modules,
we refer to the online documentation\footnote{%
\url{http://www.informatik.uni-kiel.de/~pakcs/lib/FlatCurry.Types.html} and
\url{http://www.informatik.uni-kiel.de/~pakcs/lib/AbstractCurry.Types.html}}
of these modules.

As an example, consider a program file \ccode{test.curry}
containing the following two lines:
\begin{curry}
rev []     = []
rev (x:xs) = (rev xs) ++ [x]
\end{curry}
Then the I/O action \code{(FlatCurry.Files.readFlatCurry "test")} returns the
following term:
\begin{curry}
 (Prog "test"
  ["Prelude"]
  []
  [Func ("test","rev") 1 Public
        (FuncType (TCons ("Prelude","[]") [(TVar 0)])
                  (TCons ("Prelude","[]") [(TVar 0)]))
        (Rule [0]
           (Case Flex (Var 1)
              [Branch (Pattern ("Prelude","[]") [])
                  (Comb ConsCall ("Prelude","[]") []),
               Branch (Pattern ("Prelude",":") [2,3])
                  (Comb FuncCall ("Prelude","++")
                        [Comb FuncCall ("test","rev") [Var 3],
                         Comb ConsCall ("Prelude",":")
                              [Var 2,Comb ConsCall ("Prelude","[]") []]
                        ])
              ]))]
  []
 )
\end{curry}


%%%%%%%%%%%%%%%%%%%%%%%%%%%%%%%%%%%%%%%%%%%%%%%%%%%%%%%%%%%%%%%%%%%%%%%%%
% Definitions in order to LaTeX documents generated by "currydoc --tex"
%%%%%%%%%%%%%%%%%%%%%%%%%%%%%%%%%%%%%%%%%%%%%%%%%%%%%%%%%%%%%%%%%%%%%%%%%

\newcommand{\currymodule}[1]{\subsubsection{Library #1}\label{Library:#1}}
\newcommand{\currytypesstart}{\subsubsection*{Exported types:}}
\newcommand{\currytypesstop}{}
\newcommand{\currytypesynstart}[2]{{\tt type #2}\pindex{#1} \begin{quote}}
\newcommand{\currytypesynstop}{\end{quote}}
\newcommand{\currydatastart}[1]{{\tt data #1}\pindex{#1} \begin{quote}}
\newcommand{\currydatacons}{\end{quote}%
\begin{itemize}\item[] \hspace{-4ex}\emph{Exported constructors:}}
\newcommand{\currydatastop}{\end{itemize}}
\newcommand{\curryconsstart}[2]{\item {\tt #1~::~#2}\par}
\newcommand{\curryfuncstart}{\subsubsection*{Exported functions:}}
\newcommand{\curryfuncstop}{}
\newcommand{\curryfunctionstart}[2]{#2\pindex{#1}\begin{quote}}
\newcommand{\curryfunctionstop}{\end{quote}}
\newcommand{\curryfuncsig}[2]{{\tt #1~::~#2}}


\subsection{General Libraries}

\input{lib/AllSolutions}
\input{lib/Assertion}
\input{lib/Char}
\input{lib/CLP.FD}
\input{lib/CLPFD}
\input{lib/CLPR}
\input{lib/Combinatorial}
\input{lib/CPNS}
\input{lib/CSV}
\input{lib/Debug}
\input{lib/Directory}
\input{lib/Distribution}
\input{lib/Either}
\input{lib/ErrorState}
\input{lib/FileGoodies}
\input{lib/FilePath}
\input{lib/Findall}
\input{lib/Float}
\input{lib/Function}
\input{lib/FunctionInversion}
\input{lib/GetOpt}
\input{lib/Global}
\input{lib/GlobalVariable}
\input{lib/GUI}
\input{lib/Integer}
\input{lib/IO}
\input{lib/IOExts}
\input{lib/List}
\input{lib/Maybe}
\input{lib/NamedSocket}
\input{lib/Nat}
\input{lib/Parser}
\input{lib/Ports}
\input{lib/Pretty}
\input{lib/Profile}
\input{lib/PropertyFile}
\input{lib/Read}
\input{lib/ReadNumeric}
\input{lib/ReadShowTerm}
\input{lib/SetFunctions}
\input{lib/Socket}
\input{lib/State}
\input{lib/System}
\input{lib/Time}
\input{lib/Unsafe}
\input{lib/Test.EasyCheck}
\input{lib/Test.Prop}

\subsection{Data Structures and Algorithms}

\input{lib/Array}
\input{lib/Dequeue}
\input{lib/FiniteMap}
\input{lib/GraphInductive}
\input{lib/Random}
\input{lib/RedBlackTree}
\input{lib/SCC}
\input{lib/SearchTree}
\input{lib/SearchTreeTraversal}
\input{lib/SetRBT}
\input{lib/Sort}
\input{lib/TableRBT}
\input{lib/Traversal}
\input{lib/ValueSequence}

\subsection{Libraries for Database Access and Manipulation}

\input{lib/Database.CDBI.Connection}
\input{lib/Database.CDBI.Criteria}
\input{lib/Database.CDBI.Description}
\input{lib/Database.CDBI.ER}
\input{lib/Database.CDBI.QueryTypes}
\input{lib/Database.ERD}
\input{lib/Database.ERDGoodies}
\input{lib/Database.KeyDatabaseSQLite}


\subsection{Libraries for Web Applications}

\input{lib/Bootstrap3Style}
\input{lib/HTML}
\input{lib/HtmlCgi}
\input{lib/URL}
\input{lib/WUI}
\input{lib/XML}
\input{lib/XmlConv}

\subsection{Libraries for Meta-Programming}

\input{lib/AbstractCurry.Types}
\input{lib/AbstractCurry.Files}
\input{lib/AbstractCurry.Select}
\input{lib/AbstractCurry.Build}
\input{lib/AbstractCurry.Pretty}
\input{lib/FlatCurry.Types}
\input{lib/FlatCurry.Files}
\input{lib/FlatCurry.Goodies}
\input{lib/FlatCurry.Pretty}
\input{lib/FlatCurry.Read}
\input{lib/FlatCurry.Show}
\input{lib/FlatCurry.XML}
\input{lib/FlatCurry.FlexRigid}
\input{lib/FlatCurry.Compact}

} % end setlength parindent


%%% Local Variables: 
%%% mode: latex
%%% TeX-master: "manual"
%%% End: 
