\section{Extensions}
\label{sec-extensions}

\CYS supports some extensions in Curry programs that are not (yet)
part of the definition of Curry. These extensions are described below.

\subsection{Recursive Variable Bindings}

Local variable declarations (introduced by \code{let}\pindex{let}
or \code{where}\pindex{where}) can be (mutually) recursive in \CYS.
For instance, the declaration
\begin{curry}
ones5 = let ones = 1 : ones
         in take 5 ones
\end{curry}
introduces the local variable \code{ones} which is bound
to a \emph{cyclic structure}\index{cyclic structure}
representing an infinite list of \code{1}'s.
Similarly, the definition
\begin{curry}
onetwo n = take n one2
 where
   one2 = 1 : two1
   two1 = 2 : one2
\end{curry}
introduces a local variables \code{one2} that represents
an infinite list of alternating \code{1}'s and \code{2}'s
so that the expression \code{(onetwo 6)} evaluates to \code{[1,2,1,2,1,2]}.


\subsection{Functional Patterns}

Functional patterns \cite{AntoyHanus05LOPSTR} are a useful extension
to code operations in a more readable way. Furthermore,
defining operations with functional patterns avoids problems
caused by strict equality (\ccode{=:=}) and leads to programs
that are potentially more efficient.

Consider the definition of an operation to compute the last element
of a list \code{xs} based on the prelude operation \ccode{++}
for list concatenation:
\begin{curry}
last xs | _++[y] =:= xs  = y   where y free
\end{curry}
Since the equality constraint \ccode{=:=} evaluates both sides
to a constructor term, all elements of the list \code{xs} are
fully evaluated in order to satisfy the constraint.

Functional patterns can help to improve this computational behavior.
A \emph{functional pattern}\index{functional pattern}\index{pattern!functional}
is a function call at a pattern position. With functional patterns,
we can define the operation \code{last} as follows:
\begin{curry}
last (_++[y]) = y
\end{curry}
This definition is not only more compact but also avoids the complete
evaluation of the list elements: since a functional pattern is considered
as an abbreviation for the set of constructor terms obtained by all
evaluations of the functional pattern to normal form (see
\cite{AntoyHanus05LOPSTR} for an exact definition), the previous
definition is conceptually equivalent to the set of rules
\begin{curry}
last [y] = y
last [_,y] = y
last [_,_,y] = y
$\ldots$
\end{curry}
which shows that the evaluation of the list elements is not demanded
by the functional pattern.

In general, a pattern of the form \code{($f$ $t_1$\ldots$t_n$)} ($n>0$)
is interpreted as a functional pattern if $f$ is not a visible constructor
but a defined function that is visible in the scope of the pattern.

It is also possible to combine functional patterns with
as-patterns.\index{as-pattern}\pindex{"@}
Similarly to the meaning of as-patterns
in standard constructor patterns,
as-patterns in functional patterns are interpreted
as a sequence of pattern matching where the variable of the as-pattern
is matched before the given pattern is matched.
This process can be described by introducing an auxiliary operation
for this two-level pattern matching process.
For instance, the definition
\begin{curry}
f (_ ++ x@[(42,_)] ++ _) = x
\end{curry}
is considered as syntactic sugar for the expanded definition
\begin{curry}
f (_ ++ x ++ _) = f' x
 where
  f' [(42,_)] = x
\end{curry}
However, as-patterns are usually implemented
in a more efficient way without introducing auxiliary operations.


\paragraph{Optimization of programs containing functional patterns.}
Since functions patterns can evaluate to non-linear constructor terms,
they are dynamically checked for multiple occurrences of
variables which are, if present, replaced by equality constraints
so that the constructor term is always linear
(see \cite{AntoyHanus05LOPSTR} for details).
Since these dynamic checks are costly and not necessary for
functional patterns that are guaranteed to evaluate to linear terms,
there is an optimizer for functional patterns that checks
for occurrences of functional patterns that evaluate always to
linear constructor terms and replace such occurrences
with a more efficient implementation.
This optimizer can be enabled by the following possibilities:
\begin{itemize}
\item
Set the environment variable \code{FCYPP} to \ccode{--fpopt}
before starting \CYS, e.g., by the shell command
\begin{curry}
export FCYPP="--fpopt"
\end{curry}
Then the functional pattern optimization is applied if programs are compiled
and loaded in \CYS.
\item
Put an option into the source code:
If the source code of a program
contains a line with a comment of the form (the comment
must start at the beginning of the line)
\begin{curry}
{-# PAKCS_OPTION_FCYPP --fpopt #-}
\end{curry}
then the functional pattern optimization is applied
if this program is compiled and loaded in \CYS.
\end{itemize}
The optimizer also report errors in case of wrong uses of functional patterns
(i.e., in case of a function $f$ defined with functional patterns that
recursively depend on $f$).


\subsection{Order of Pattern Matching}

Curry allows multiple occurrences of pattern variables
in standard patterns. These are an abbreviation of equational constraints
between pattern variables.
Functional patterns might also contain multiple occurrences of
pattern variables.
For instance, the operation
\begin{curry}
f (_++[x]++_++[x]++_) = x
\end{curry}
returns all elements with at least two occurrences in a list.

If functional patterns as well as multiple occurrences of
pattern variables occur in a pattern defining an operation,
there are various orders to match an expression against such
an operation. In the current implementation, the order
is as follows:
\begin{enumerate}
\item Standard pattern matching: First, it is checked whether
the constructor patterns match. Thus, functional patterns
and multiple occurrences of pattern variables are ignored.
\item Functional pattern matching: In the next phase,
functional patterns are matched but occurrences of standard
pattern variables in the functional patterns are ignored.
\item Non-linear patterns: If standard and functional pattern matching
is successful, the equational constraints which correspond
to multiple occurrences pattern variables are solved.
\item Guards: Finally, the guards supplied by the programmer
are checked.
\end{enumerate}
The order of pattern matching should not influence the computed
result. However, it might have some influence on the termination
behavior of programs, i.e., a program might not terminate
instead of finitely failing.
In such cases, it could be necessary to consider the influence
of the order of pattern matching. Note that other orders of pattern matching
can be obtained using auxiliary operations.


\subsection {Datatypes with Field Labels}\label{flab}

A datatype declaration may optionally define data constructors
with field labels.\footnote{Field labels are quite similar
to Haskell \cite{PeytonJones03Haskell} so that we adopt most of
the description of Haskell here.}
These field labels can be used to
construct, select from, and update fields in a manner that is
independent of the overall structure of the datatype.

\subsubsection{Declaration of Constructors with Labeled Fields}

A data constructor of arity $n$ creates an object with $n$ components.
These components are normally accessed positionally as arguments to
the constructor in expressions or patterns.  For large datatypes it is
useful to assign \emph{field labels}\index{label}\index{field label}
to the components of a data
object. This allows a specific field to be referenced independently of
its location within the constructor.  A constructor definition in a
data declaration may assign labels to the fields of the constructor,
using the \emph{record syntax}\index{record syntax}\pindex{\{\ldots{}\}}
\code{C\,\,\{}\ldots\code{\}}.
Constructors using field labels
may be freely mixed with constructors without them. A constructor with
associated field labels may still be used as an ordinary constructor.
The various use of labels (see below)
are simply a shorthand for operations using an
underlying positional constructor. The arguments to the positional
constructor occur in the same order as the labeled fields.

\newcommand{\trans}[1]{$[\![$#1$]\!]$}

\translation{\\
\trans{$C$ \{ $lts$ \}} $=$ $C$ \trans{$lts$}\\
\trans{$lt$, $lts$} $=$ \trans{$lt$} \trans{$lts$}\\
\trans{$l$, $ls$\,::\,$t$} $=$ $t$ \trans{$ls$\,::\,$t$}\\
\trans{$l$\,::\,$t$} $=$ $t$
}

\noindent
For example, the definition using field labels
\begin{curry}
data Person = Person {firstName, lastName :: String, age :: Int}
            | Agent { firstName, lastName :: String, trueIdentity :: Person }
\end{curry}
is translated to
\begin{curry}
data Person = Person String String Int
            | Agent String String Person
\end{curry}
%
A data declaration may use the same field label in multiple
constructors as long as the typing of the field is the same in all
cases after type synonym expansion. A label cannot be shared by more
than one type in scope. Field names share the top-level name space with
ordinary definition of functions % and class methods
and must not conflict with other top-level names in scope.

Consider the following example:
\begin{curry}
data S = S1 { x :: Int } | S2 { x :: Int  } -- OK
data T = T1 { y :: Int } | T2 { y :: Bool } -- BAD
\end{curry}
Here S is legal but T is not, because y is given inconsistent typings in the latter.

\subsubsection{Field Selection}\label{flab-sel}
Field labels are used as \emph{selector functions},\index{selector function}
i.e., each field label serves as a function
that extracts the field from an object. Selectors are top-level
bindings and so they may be shadowed by local variables but cannot
conflict with other top-level bindings of the same name.  This
shadowing only affects selector functions; in record construction
(Section \ref{flab-constr}) and update (Section \ref{flab-upd}), field
labels cannot be confused with ordinary variables.

\translation{A field label $lab$ introduces a selector function defined as:\\
\code{$lab$ ($C_1$ $p_{11}$ \ldots $p_{1k_1}$) = x}\\
\ldots\\
\code{$lab$ ($C_n$ $p_{n1}$ \ldots $p_{nk_n}$) = x}\\
where $C_1\ldots C_n$ are all the constructors of the datatype containing
a field labeled with $lab$, $p_{ij}$ is \code{x} when $lab$ labels the $j$th
component of $C_i$ or \code{\us} otherwise.}

\noindent
For example the definition of \code{Person} above introduces the selector functions
\begin{curry}
firstName :: Person -> String
firstName (Person x _ _) = x
firstName (Agent  x _ _) = x

lastName :: Person -> String
lastName (Person _ x _) = x
lastName (Agent  _ x _) = x

age :: Person -> Int
age (Person _ _ x) = x

trueIdentity :: Person -> Person
trueIdentity (Agent  _ _ x) = x
\end{curry}

\subsubsection{Construction Using Field Labels}\label{flab-constr}

A constructor with labeled fields may be used to construct a value in
which the components are specified by name rather than by
position. In this case, the components are enclosed by braces.
Construction using field labels is subject to the following
constraints:
%
\begin{itemize}
\item Only field labels declared with the specified constructor may be
  mentioned.
\item A field label may not be mentioned more than once.
\item Fields not mentioned are initialized to different free variables.
\end{itemize}
%
The expression \code{C\{\}}, where \code{C} is a data constructor,
is legal \emph{whether or not  \code{C} was declared with record syntax};
it denotes \code{C (let x free in x)$_1$ \ldots{} (let x free in x)$_n$}
where $n$ is the arity of \code{C}.
Note that this will introduce the constructor \code{C} with
$n$ \emph{different} free variables as arguments.

\translation{In the binding $f = v$, the field $f$ labels $v$.\\
\centerline{\code{$C$ \{ $bs$ \} = $C$ ($pick_1^C\ bs\ \code{(let x free in x)}$) \ldots{}
($pick_k^C\ bs\ \code{(let x free in x)}$)}}
where $k$ is the arity of $C$.\\
The auxiliary function $pick_i^C\ bs\ d$ is defined as follows:

If the $i$th component of a constructor $C$ has the field label $f$ and $f = v$
appears in the binding list $bs$, then $pick_i^C\ bs\ d$ is $v$.
Otherwise, $pick_i^C\ bs\ d$ is the default value $d$.}

\noindent
For example, a \code{Person} can be constructed by
\begin{curry}
smith = Agent {lastName="Smith", firstName="Agent"}
\end{curry}
which is equivalent to the following agent, whose true identity
might be any person:
\begin{curry}
smith = Agent "Agent" "Smith" (let x free in x)
\end{curry}

\subsubsection{Updates Using Field Labels}\label{flab-upd}

Values belonging to a datatype with field labels may be
non-destructively updated. This creates a new value in which the
specified field values replace those in the existing value. Updates
are restricted in the following ways:

\begin{itemize}
\item All labels must be taken from the same datatype.
\item No label may be mentioned more than once.
\item The computation fails when the value being updated does not contain all of the specified
labels.
\end{itemize}

\translation{Using the prior definition of $pick$,\\
\code{$e$ \{ $bs$ \} $=$ let $upd$ ($C_1$ $v_1$\ldots{}$v_{k_1}$) = $C_1$ ($pick_1^{C_1}$ $bs$ $v_1$) \ldots{} ($pick_{k_1}^{C_1}$ $bs$ $v_{k_1}$)}\\
\hspace*{2.7cm}\ldots\\
\hspace*{2.7cm}\code{$upd$ ($C_j$ $v_1$\ldots{}$v_{k_j}$) = $C_j$ ($pick_1^{C_j}$ $bs$ $v_1$) \ldots{} ($pick_{k_j}^{C_j}$ $bs$ $v_{k_j}$)}\\
\hspace*{2.23cm}\code{in $upd~e$}\\[1ex]
where $\{C_1,\ldots,C_j\}$ is the set of constructors containing \emph{all} labels
in $bs$, $k_i$ is the arity of $C_i$, and $upd$ is a new function name.}

\noindent
For example, after watching a few more movies, we might want to update
our information about \code{smith}. We can do so by writing
\code{smith\{trueIdentity=complement neo\}}, which is equivalent to
\begin{curry}
let upd (Agent fn ln ti) = Agent (fn) (ln) (complement neo)
 in upd smith
\end{curry}

\subsubsection{Pattern Matching Using Field Labels}\label{flab-pat}

A constructor with labeled fields may be used to specify a pattern in
which the components are identified by name rather than by position.
Matching against a constructor using labeled fields is the same as
matching ordinary constructor patterns except that the fields are
matched in the order they are named in the field list. All listed
fields must be declared by the constructor; fields may not be named
more than once. Fields not named by the pattern are ignored (matched
against \code{\us}).

\translation{Using the prior definition of $pick$,\\
\code{$C$ \{ bs \} = ($C$ ($pick_1^C$ $bs$ _) \ldots{} ($pick_{k}^C$ $bs$ _))}\\
where $k$ is the arity of $C$.}

\noindent
For example, we could define a Smith-tester by writing:
\begin{curry}
isSmith Agent{lastName="Smith"} = success
\end{curry}
which is equivalent to
\begin{curry}
isSmith (Agent _ "Smith" _) = success
\end{curry}

\subsubsection{Field Labels and Modules}

As described in Curry report, there are two forms of
exporting a data type $T$: The simple name $T$ exports only the types name
without constructors, whereas $T$\code{(..)} also exports all
constructors. Analogously, the form $T$ does not export any field labels,
whereas $T$\code{(..)} exports all constructors and all field labels.

There is the additional possibility of exporting the selector functions which
are implicitly introduced by declaring a constructor with field labels,
(see Section~\ref{flab-sel}).
In contrast to \cite{PeytonJones03Haskell}, this does
not enable the possibility of updating a value using the field label syntax
(see Section~\ref{flab-upd}). This is prohibited mainly for two reasons:
\begin{itemize}
\item The translations given in this section are still valid in the context of modules.
\item From a software engineering point of view, it can make sense to
  export selectors for a certain type without allowing updates on it.
  Changing a given value is more often constrained by integrity
  properties than retrieving partial information from it.
\end{itemize}


%%% Local Variables: 
%%% mode: latex
%%% TeX-master: "manual"
%%% End: 


%%% Local Variables: 
%%% mode: latex
%%% TeX-master: "manual"
%%% End: 
